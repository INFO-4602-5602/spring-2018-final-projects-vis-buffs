\documentclass{vgtc}

\ifpdf%                                % if we use pdflatex
\pdfoutput=1\relax                   % create PDFs from pdfLaTeX
\pdfcompresslevel=9                  % PDF Compression
\pdfoptionpdfminorversion=7          % create PDF 1.7
\ExecuteOptions{pdftex}
\usepackage{graphicx}                % allow us to embed graphics files
\DeclareGraphicsExtensions{.pdf,.png,.jpg,.jpeg} % for pdflatex we expect .pdf, .png, or .jpg files
\else%                                 % else we use pure latex
\ExecuteOptions{dvips}
\usepackage{graphicx}                % allow us to embed graphics files
\DeclareGraphicsExtensions{.eps}     % for pure latex we expect eps files
\fi%

\usepackage{booktabs} % For formal tables
\usepackage{color, soul}
\usepackage{hyperref}
\usepackage{url}
\def\UrlBreaks{\do\/\do-}
\graphicspath{{figures/}}
\usepackage{amsthm}
\theoremstyle{definition}
\newtheorem{definition}{Definition}[section]
\usepackage{subcaption}
\usepackage{graphicx}
\usepackage[]{units}

\abstract{Yelp is a crowd-sourced local business review and social networking community, which has hundreds of thousands of users contribute their data every day. Based on users' reviews and ratings, good local businesses stand out among their categories on top of the list, acting as a word-of-mouth reference. However, tons of user data doesn't make any sense unless we make good use of it. In this case, we decided to look into the problem that how we might get as much as useful information from our data with a particular interest in how the users' rating behavior are influenced by different factors, and what kind of prediction we can make out of the rating pattern we found. With various features extracted from Yelp data, we conducted feature selection, best split and finally built a Decision Tree model which predicts users' rating based on those features. Our visualization includes the complete process of this typical machine learning method, which provides insights about the inner mechanism of how the rating prediction is conducted. }

\begin{document}
\title{Yelp Rating Prediction Visualization}

\author{Jiawen Liu\thanks{e-mail: Jiawen Liu}\\ %
	\and Keke Wu\thanks{e-mail: Keke.Wu@colorado.edu}\\ %
	\and Wei Miao\thanks{e-mail: Wei.Miao@colorado.edu}\\ %
	\and Xu Han\thanks{e-mail: Xu.Han@colorado.edu}\\ %
	\and Yawen Zhang\thanks{e-mail: Yawen.Zhang@colorado.edu}\\ %
	}

%\author{Jiawen Liu}
%\email{Jiawen Liu}
%
%\author{Keke Wu}
%\email{Keke.Wu@colorado.edu}
%
%\author{Wei Miao}
%\email{Wei.Miao@colorado.edu}
%
%\author{Xu Han}
%\email{Xu.Han@colorado.edu}
%
%\author{Yawen Zhang}
%\email{Yawen.Zhang@colorado.edu}

%\author{Charles Palmer}
%\affiliation{%
%  \institution{Palmer Research Laboratories}
%  \streetaddress{8600 Datapoint Drive}
%  \city{San Antonio}
%  \state{Texas} 
%  \postcode{78229}}
%\email{cpalmer@prl.com}

%\begin{abstract}

\textcolor{red}{Your motivating problem, what you did, and what you found}

Yelp is a crowd-sourced local business review and social networking community, which has hundreds of thousands of users contribute their data every day. Based on users' reviews and ratings, good local businesses stand out among their categories on top of the list, acting as a word-of-mouth reference. However, tons of user data doesn't make any sense unless we make good use of it. In this case, we decided to look into the problem that how we might get as much as useful information from our data with a particular interest in how the users' rating behavior are influenced by different data attributes, and what kind of prediction we can make out of the rating pattern we found.


\end{abstract}

 

%\keywords{Yelp; Rating Prediction; Decision Tree}

\maketitle

\section{Introduction}
\label{sec:intro}

%\textcolor{red}{The motivating Problem, why it's interesting or important}

Rating prediction plays an important role in the recommendation system as to capture users' preference for specific product. In Yelp, user can post their rating and review for a restaurant they visited, and these data as well as user/restaurant related attributes, e.g., user average rating, restaurant location or category, can be well gathered for conducting the rating prediction. If a user's rating for a restaurant can be accurately predicted, Yelp would be able to recommend high rating product to the user which meets their preferences. 

The challenge in this problem lies in the feature selection and modeling processing. With multiple features related to a user's rating, picking up the effective features is quite important, which would significantly influence the prediction performance. Additionally, the modeling process which selects the best machine learning model for rating prediction is also critical, and it's necessary to gain insights about how the model functions, i.e., the inner mechanism of model. Previously, most machine learning processes have been conducted in a "black box", by just showing input and output. In our project, we aim at visualizing the whole process, including feature selection, best split, and model tree. The Decision Tree model is selected because its straightforward idea in classification, which would be beneficial for visualization. 

\section{Related Work}
\label{sec:related} 

%\textcolor{red}{Summarize research related to your projects, minimum 8 citations. }

In this project, we deploy the visualization principles and techniques to make mechanism of the whole recommendation system(RS) transparent based on Yelp's public dataset\cite{}. A lot of research has been done on recommendation system and the RS techniques are broadly divided into two types: memory-based approach, which recommend business based on similarity or correlation between users\cite{}, and model-based approach, which use machine learning methods to predict user ratings\cite{}. In our project, we use decision tree from model-based approach\cite{} as our visualization example. 

Our visualization of modeling process mainly focus on four parts: feature engineering, best split analysis, feature ranking and model training. For feature engineering, we extract 22 features in total, includes user-related features(7), business-related features(3), user-category features(5) and review-related features(7). Among review-related features, we extract several advanced natural language processing(NLP) features like polarity\cite{} and subjectivity\cite{}. For best spllit analysis, we create a moveable threshold to study how this feature -- business average star, could influence the decision making(whether to recommend or not). When moving the threshold, the calculated accuracy and true positive rate\cite{} will be changed correspondingly and we can choose the threshold with highest accuracy as our decision tree's best split. In the part of feature ranking, we measure the importance of all 22 features based on the score retrieved by Xgboost\cite{}. Xgboost's feature importance method calculateds F score, which indicates how many times the feature split on. Higher the F score is, more important the feature is. In our feature ranking visualization, we use a radar graph to show the importance of all these features based on F score. The last step is model training, we use 100 users as an example. 80 users are used to train and 20 users are used to test. The top three features with highest F score are selected and used by the model. We visualize each users path and the overall test accuracy.
\section{Design Process}
\label{sec:design} 

\textcolor{red}{Justifications for any design elements}
\section{Modeling Process}
\label{sec:modeling} 

\textcolor{red}{Decision Tree}
\section{Results}
\label{sec:results} 

Our visualization storyline include four parts, features, best split, feature ranking and tree model. 

\subsection{Vis 1: Features}

In this part, as shown in Figure \ref{fig:vis_1}, we selected four typical features related to users' rating, including 1) restaurant type, visualized with a bar chart, 2) restaurant location, visualized with a map, 3) frequent words related to different ratings, visualized with a word cloud, 4) friendship network, visualized with a network. With these four selected features and different visualizations, we aimed to illustrate the difference of features. In each visualization, we added the rating information by either using the tooltip or other methods to show how the features are related to user's rating. 

%-------------------------------------------
\begin{figure}[h]
	\centering
	\includegraphics[width=9cm]{vis_1.png}
	\caption{Visualizations of selected features related to users' rating.}
	\label{fig:vis_1}
\end{figure}
%------------------------------------------- 

\subsection{Vis 2: Best Split}

As to demonstrate how different cut-offs would influence the prediction results, we used user's average rating as an example, to illustrate the influence of thresholds on correct and other rates in the prediction results. The best split is determined by choosing the right threshold of the feature. As shown in Figure \ref{fig:vis_2}, when the threshold of user average rating is changed, the correct, incorrect, true positive, positive rate would all change. For all the features used for rating prediction, their best splits would be determined in advance. 

%-------------------------------------------
\begin{figure}[h]
	\centering
	\includegraphics[width=9cm]{vis_2.png}
	\caption{Visualizations of how to choose best split.}
	\label{fig:vis_2}
\end{figure}
%------------------------------------------- 

\subsection{Vis 3: Feature Ranking}

As shown in Figure \ref{fig:vis_3}, with the feature ranking results shown in Table \ref{tb:ranking}, we further visualize the result using an radar map. This map illustrate the importance of different features in rating prediction. This is an important process before generating the tree model as only high ranked features would be used in the tree model. 

%-------------------------------------------
\begin{figure}[h]
	\centering
	\includegraphics[width=9cm]{vis_3.png}
	\caption{Visualizations of feature ranking result.}
	\label{fig:vis_3}
\end{figure}
%------------------------------------------- 

\subsection{Vis 4: Tree Model}

Finally, with selected features from ranking, we built the decision tree model for rating prediction as shown in Figure \ref{fig:vis_4}. This visualization include the basic tree model with nodes represent the class labels and edges represent conjunctions of feature that lead to the class labels. This tree map works in a dynamic way of showing how a given input would be classified, i.e., how a user's rating would be predicted given related features. Additionally, the training and test accuracy are also calculated. 

%-------------------------------------------
\begin{figure}[h]
	\centering
	\includegraphics[width=9cm]{vis_4.png}
	\caption{Visualizations of tree model and training/test accuracy.}
	\label{fig:vis_4}
\end{figure}
%-------------------------------------------

\section{Discussion}
\label{sec:discussion} 

In the project, we conducted a visualization of Yelp rating prediction. This visualization not only visualize the data but also visualize the machine learning process, as to reveal information in the "black box". With four parts of visualization, 1) features, 2) best split, 3) feature ranking, 4) tree model, we can understand the Decision Tree model better and gain insights of each part through visualizations. 

\bibliographystyle{ACM-Reference-Format}
\bibliography{main} 

\end{document}
