\section{Design Process}
\label{sec:design} 

\textcolor{red}{Justifications for any design elements}

During the design process, we tried to get our results displayed in a reader-driven rather than author-driven way, following a storytelling style. Our preliminary works included data cleaning, data organization by categories, feature selection, hand-written sketches of the complete workflow. We used web as our platform and D3.js as the main tool. The web was divided into six different but consistent sections in the order of Our Story, Our Goal, Our Design, Visualization, Split and Tree Model, which was exactly the sequence we were following to dive into this topic. In terms of the visual design, we used the red, black and white color scheme adopted by Yelp to make it look consistent. Besides text descriptions and interactive visualization graphs, we also had hand-written sketches on the web as a proof of concept, which helped enrich the design elements' diversity. In terms of the visualization design, we used four different types of charts to visualize four different rating related data features. The restaurant ratings by category were visualized in a bar chart, showing customers' preferences on cuisines. The average ratings by state were visualized in a US map, with the darker colored state having a higher rating trend. We also extracted key words from users' reviews and got them displayed in a randomly generated word cloud, where the words were extinguished from each other by five colors mapped with five different rating stars. Besides these, we also built a network to visualize the influence on ratings from Yelp users who have most friends. All of these charts were interactive and dynamic, enabling a good user-directed exploration experience. Following sections utilized a split framework to explain how we applied machine learning to help us make predictions. Finally, a decision-tree like model walked the audience through the decision and prediction making process. With a well-designed and fully-interactive design style, we hope to help our audience get valuable information in a way that can be adjusted as they like. Meanwhile, we hope to use some vivid graphs and animations to bridge the gap between scientific research and general public understanding, making the abstract concept intuitive and straightforward.
